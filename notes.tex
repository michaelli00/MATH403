\documentclass{article}
\usepackage[top=1in,bottom=1in,left=0.5in,right=0.5in]{geometry}
\usepackage{hyperref}
\usepackage{amsmath}
\usepackage{amssymb}
\usepackage{graphicx}
\usepackage{parskip}
\graphicspath{ {./assets/} }
\usepackage{listings}
\usepackage{color}
\definecolor{mygreen}{rgb}{0,0.6,0}
\definecolor{mygray}{rgb}{0.5,0.5,0.5}
\definecolor{mymauve}{rgb}{0.58,0,0.82}
\lstset{ 
  backgroundcolor=\color{white},   % choose the background color; you must add \usepackage{color} or \usepackage{xcolor}; should come as last argument
  basicstyle=\footnotesize,        % the size of the fonts that are used for the code
  breakatwhitespace=false,         % sets if automatic breaks should only happen at whitespace
  breaklines=true,                 % sets automatic line breaking
  captionpos=b,                    % sets the caption-position to bottom
  commentstyle=\color{mygreen},    % comment style
  escapeinside={\%*}{*)},          % if you want to add LaTeX within your code
  firstnumber=0,                   % start line enumeration with line 0
  frame=single,	                   % adds a frame around the code
  keepspaces=true,                 % keeps spaces in text, useful for keeping indentation of code (possibly needs columns=flexible)
  keywordstyle=\color{blue},       % keyword style
  language=C,                      % the language of the code
  numbers=left,                    % where to put the line-numbers; possible values are (none, left, right)
  numbersep=5pt,                   % how far the line-numbers are from the code
  numberstyle=\tiny\color{mygray}, % the style that is used for the line-numbers
  rulecolor=\color{black},         % if not set, the frame-color may be changed on line-breaks within not-black text (e.g. comments (green here))
  showspaces=false,                % show spaces everywhere adding particular underscores; it overrides 'showstringspaces'
  showstringspaces=false,          % underline spaces within strings only
  showtabs=false,                  % show tabs within strings adding particular underscores
  stepnumber=2,                    % the step between two line-numbers. If it's 1, each line will be numbered
  stringstyle=\color{mymauve},     % string literal style
  tabsize=2,	                     % sets default tabsize to 2 spaces
  title=\lstname                   % show the filename of files included with \lstinputlisting; also try caption instead of title
}
\usepackage[none]{hyphenat}
\date{}
\title{MATH403 Introduction to Abstract Algebra}
\begin{document} 
  \author{Michael Li}
  \title{MATH403 Introduction to Abstract Algebra}
  \maketitle
  \tableofcontents
  \newpage
  \section{Preliminaries}
  \textbf{Well Ordering Principle}: every nonempty set of positive integers has a smallest member

  \bigskip
  
  \textbf{Division Algorithm}: let $a, b \in Z$ with $b > 0$ then there exist unique $q, r \in Z$ such that $a = bq + r$ where $0 \leq r < b$

  \textbf{Proof}: broken down into existence and uniqueness
  \begin{itemize}
    \item Existence: consider the set $S = \{a - bk \mid k \in Z \wedge a - bk \geq 0\}$
      \begin{itemize}
        \item if $0 \in S$ then $b \mid a \implies q = a/b$ and $r = 0$
        \item if $0 \notin S$ then
          \begin{itemize}
            \item if $a > 0, a - b \cdot 0 \in S$
            \item if $a < 0, a - b(2a) = a(1 - 2b) \in S$ (\textbf{note}: we are dealing with integers)
          \end{itemize}
          So $S$ is nonempty. Applying Well Ordering Principle, $S$ has a smallest member $r = a - bq$ where $r \geq 0$

          To show that $r < b$, assume by contradiction that $r \geq b$, then $r - b \geq 0 \implies a - bq -b = a - b(q + 1) \in S$. 

          However, $a - b(q+1) < a - bq$ which is a contradiction since $a - bq$ is not the smallest member. Thus $r < b$.
      \end{itemize}
    \item Uniqueness: suppose $q, q', r, r' \in Z$ such that
      \[a = bq + r, \, 0 \leq r < b \quad \text{and} \quad a = bq' + r', \, 0 \leq r' < b\]
      Without loss of generality, suppose $r' \geq r$, then $bq + r = bq' + r' \implies b(q - q') = r' - r$. 

      Note that $0 \leq r' - r < b$ so the only multiple of $b$ that satisfies the inequality above is $0$

      Thus $r' = r \implies q' = q$
  \end{itemize}

  \bigskip
  
  \textbf{GCD Is a Linear Combo}: for any nonzero $a, b \in Z$, there exists $s, t \in Z$ such that $\gcd(a, b) = as + bt$. Moreover, $\gcd(a, b)$ is the smallest positive integer of the form $as + bt$

  \textbf{Proof}: broken down into existence, smallest, and greatest divisor

  \begin{itemize}
    \item Existence: consider $S = \{am + bn \mid m, n \in Z \wedge am + bn > 0\}$

      Since $S$ is nonempty, by the Well Ordering Principle, $S$ has a smallest member, say $d = as + bt$
    \item Smallest: we claim that $d = \gcd(a, b)$

    Using Division Algorithm, we have $a = dq + r$ where $0 \leq r < d$

    If $r > 0$ then $r = a - dq = a - (as + bt)q = a(1-sq) + b(-tq) \in S$ contradicting that $d$ is the smallest member of $S$

    So $r = 0$ and $d \mid a$

    Analogously, $d \mid b$ so $d$ is a common divisor of $a,b$
    \item Greatest Divisor: suppose $d'$ is another common divisor, then $a = d' h \quad \text{and} \quad b=d'k$

   Then $d = as + bt = (d'h)s + (d'k)t = d'(hs + kt)$ so $d'$ is a divisor of $d$ and $d$ is the greatest divisor
 \end{itemize}

 \textbf{GCD Corollary}: if $a,b$ are relatively prime, then $\exists s, t\in Z$ such that $as + bt = 1$

 \bigskip

 \textbf{Euclid's Lemma}: if $p$ is prime and $p \mid ab$ then $p \mid a$ or $p \mid b$

 \textbf{Proof}: Suppose $p \nmid a$, then $\gcd(p, a) = 1 \implies 1 = as + pt$ for $s, t \in Z$ since GCD can be represented as a linear combo

  This means that $b = bas + bpt$. Since $p \mid$ RHS $\implies p \mid$ LHS

  Thus $p \mid b$

  \bigskip

 \textbf{Fundamental Theorem of Arithmetic}: every integer $> 1$ is a prime or a product of primes. This product is unique.

 \textbf{Proof}: broken down into existence of a product of primes and unique product
 \begin{itemize}
   \item Existence: let $S$ be the set of positive integers that cannot be factored as a product of primes

     By the Well Ordering Principle, $S$ has a smallest member $n$

     Since $n$ is not prime, $n = ab$ where $1 < a, b < n$

     Both $a, b \notin S$ so they are a product of primes

     But $n = ab$ so $n$ is a product of primes, thus a contradiction is reached and $S$ is empty
  
   \item Uniqueness: let $S$ be a set of positive integers with non-unique prime factorizations

     By the Well Ordering Principle, $S$ has a smallest member $n = p_1 \ldots p_r = q_1 \ldots q_s$

     Note that $p_1 \mid n \implies p_1 \mid q_1 \ldots q_s$

     By Euclid's Generalized Lemma, $p_1 \mid q_j$ for some $1 \leq j \leq s$

     Since both $p_1$ and $q_j$ are prime, $p_1 = q_j$

     After reordering the $q_k$ factors, we get $p_2 \ldots p_r = q_2 \ldots q_s < n$ so $\notin S$

     Thus the remaining factors have a unique factorization and $r = s$ and $p_2 \ldots p_r$ are the same as $p_2 \ldots q_s$

     Thus $S$ is empty
 \end{itemize}
 \bigskip

 \textbf{Multiples of lcm(a,b)}: let $a,b \in Z$ be nonzero. Then every common multiple of $a,b$ is a multiple of $\lcm(a, b)$

 \textbf{Proof}: let $m = \lcm(a, b)$ and $M$ be a multiple of $a,b$. 

 By definition of lcm, $m \leq M$

 By the division algorithm, $M = mq + r$ for $q, r \in Z$ and $0 \leq r < m$

 Implies $r = M - mq$ and $ab \mid$ RHS $\implies ab \mid$ LHS

 Since $r$ is restricted to $0 \leq r < m$ and $m$ is the lowest multiple of $ab$, we have that $r = 0$

 Thus $M = qm$ and $m \mid M$

 \bigskip

 \textbf{First Principle of Mathematical Induction}: let $S$ be a set of integers containing $a$. Suppose $S$ has the property that for some integers $n \geq a, n \in S$, then $n + 1 \in S$. Then, $S$ contains every integer greater than or equal to $a$

 \textbf{Proof}: let $A$ be an nonempty set consisting of integers $n \geq a$ where $P(n)$ doesn't hold

 By the Well Ordering Principle, $A$ has a least element, call it $m$.

 Since $P(a)$ is true, $a \neq m$. Also, since $m$ is the smallest member of $A$, $P(m-1)$ is true

 But then the property holds for $(m-1) + 1$ thus a contradiction is reached and $A$ is empty

 Thus $S$ contains all integers $\geq a$

 \bigskip

 \textbf{Second Principle of Mathematical Induction}: let $S$ be a set of integers containing $a$. Suppose $S$ has the property that $n \in S$ whenever every integer $< n$ and $\geq a$ is in $S$. Then $S$ contains every integer $\geq a$

 \textbf{Proof}: let $A$ be a nonempty set consisting of integers $n \geq a$ where $P(n)$ doesn't hold

 By the Well Ordering Principle, $A$ has a smallest element, call it $m$. 

 Since $P(a)$ holds, $a \neq m$

 This means that $P(a), P(a+1), \ldots, P(m-2), P(m-1)$ hold, which implies that $P(m)$ holds and we have a contradiction

 Thus $A$ is empty and $S$ contains all integers $\geq a$

 \bigskip

 \textbf{Equivalence Relation}: an \textbf{equivalence relation} on set $S$ is a set $R$ of ordered pairs of elements of $S$ such that
 \begin{enumerate}
   \item \textbf{Reflexive Property}: $(a, a) \in R$ for all $a \in S$
   \item \textbf{Symmetric Property}: $(a, b) \in R \implies (b, a) \in R$
    \item \textbf{Transitive Property}: $(a, b) \in R \wedge (b, c) \in R \implies (a, c) \in R$
 \end{enumerate}

 \textbf{Partition}: a \textbf{partition} of set $S$ is a collection of nonempty disjoint subsets of $S$ whose union is $S$

 \textbf{Equivalence Classes Partition}: the equivalence classes of an equivalence relation on a set $S$ constitute a partition of $S$. Conversely, for any partition $P$ of $S$, there is an equivalence relation on $S$ whose equivalence classes are the elements of $P$

 \textbf{Proof}: let $\sim$ be an equivalence relation on set $S$.
 \begin{itemize}
   \item for any $a \in S$, reflexive property shows that $a \in [a]$ so $[a]$ is nonempty and the union of all equivalence classes is $S$
  \item suppose $[a]$ and $[b]$ are distinct equivalence classes, need to show that $[a] \wedge [b] = \emptyset$

    By contradiction, assume $c \in [a] \wedge [b]$

    Let $x \in [a]$ then we have $c \sim a, c \sim b, x \sim a$.

    By symmetric property, we also have that $a \sim c$ and by transitivity we have $x \sim c$ and $x \sim b$.

    Thus $[a] \subseteq [b]$. Analogously $[b] \subseteq [a]$.

    Thus $[a] = [b]$ which yields a contradiction that $[a]$ and $[b]$ were distinct equivalence classes

    Thus $[a]$ and $[b]$ are disjoint
 \end{itemize}
 To prove the converse, let $P$ be a collection of nonempty disjoint subsets of $S$ whose union is $S$.

 Define $a \sim b$ if $a,b$ belong in the same subset
 \begin{itemize}
   \item Reflexivity: since the union of the subsets form $S$ every $x \in S$ belongs to some subset
    \item Symmetry: by definition if $a, b$ are in the same subset, then $b, a$ are in the same subset
    \item Transitivity: if $a, b$ are in the subset and $b, c$ are in the same subset, then these must be the same subset since partitions must be disjoint. Thus $a, c$ are in the same subset
 \end{itemize}
  \section{Groups}
  \textbf{Binary Operation}: binary operation on set $G$ is a function that assigns each ordered pair of elements of $G$ an element of $G$
  \begin{itemize}
    \item this preserves \textbf{closure}, meaning that the members of an ordered pair from $G$ yield a member of $G$
  \end{itemize}

  \textbf{Group}: let $G$ be a set together with a binary operation that assigns each ordered pair $(a,b)$ of elements of $G$ an element in $G$, denoted $ab$. $G$ is a \textbf{group} if all 3 are satisfied:
  \begin{enumerate}
    \item \textbf{Associativity}: operation is associative so $(ab)c = a(bc)$ for all $a, b, c \in G$
    \item \textbf{Identity}: there is an \textbf{identity element} $e \in G$ such that $ae = ea = a$ for all $a \in G$
    \item \textbf{Inverses}: for each element $a \in G$ there is an \textbf{inverse element} $b \in G$ such that $ab = ba = e$
  \end{enumerate}

  \textbf{Abelian (commutative)}: a group is Abelian if for every pair of elements $a,b$ we have $ab = ba$. Otherwise it is non-Abelian if there is some pair of elements $a,b$ such that $ab \neq ba$

  \textbf{Examples}:
  \begin{enumerate}
    \item set of integers $Z$, rational numbers $Q$, and real numbers $R$ are groups under ordinary addition
      \begin{itemize}
        \item associativity is held
        \item identity is $0$
        \item inverse of $a$ is $-a$
      \end{itemize}
    \item set of integers under ordinary multiplication is NOT a group
      \begin{itemize}
        \item there is no integer $b$ such that $5b = 1$
      \end{itemize}
    \item subset $\{1, -1, i , -i\}$ of complex numbers is a group under complex multiplication
      \begin{itemize}
        \item associativity is held
        \item identity is $1$
        \item all terms have an inverse that exists in the subset
      \end{itemize}
    \item set $Q^+$ is a group under ordinary multiplication
      \begin{itemize}
        \item associativity is held
        \item identity is $1$
        \item inverse of any $a$ is $1/a = a^{-1}$
      \end{itemize}
    \item set $S$ of positive irrational numbers and $1$, although it satisfies the 3 given properties, it is not a group
      \begin{itemize}
        $\sqrt{2} \cdot \sqrt{2} = 2 \notin S$ so $S$ is not closed under multiplication.
      \end{itemize}
    \item rectangular matrix $\begin{bmatrix} a & b \\ c & d \end{bmatrix}$ of real entries is a group under componentwise addition
      \begin{itemize}
        \item associativity is held
        \item identity is $\begin{bmatrix} 0 & 0 \\ 0 & 0 \end{bmatrix}$
        \item inverse of $\begin{bmatrix} a & b \\ c & d \end{bmatrix}$ is $\begin{bmatrix} -a & -b \\ -c & -d \end{bmatrix}$
      \end{itemize}
    \item $Z_n = \{0, 1, \ldots, n - 1\}$ for $n \geq 1$ is a group under addition modulo $n$
      \begin{itemize}
        \item associativity is held
        \item identity is $0$
        \item for $j > 0 \in Z_n$, inverse of $j$ is $n - j$
      \end{itemize}
  \end{enumerate}
  \subsection{Elementary Properties of Groups}
  \textbf{Theorem 2.1 Uniqueness of the Identity}: in a group $G$, there is only 1 identity element
    
    \textbf{Proof}: suppose $e$ and $e'$ are both identities of $G$. Then
    \begin{enumerate}
      \item $ae = a$ for all $a \in G$ and
      \item $e'a = a$ for all $a \in G$
    \end{enumerate}

    Then $e'e = e'$ and $e'e = e$ so $e' = e$

    \bigskip

    \textbf{Theorem 2.2 Cancellation}: in a group $G$, the right and left cancellation laws hold. That is
      \[ba = ca \implies b =c \quad \text{ and } \quad ab = ac \implies b = c\]
      \textbf{Proof}: suppose $ba = ca$ and let $a'$ be the inverse of $a$
      \[(ba)a' = (ca)a' \implies b(aa') = c(aa') \, \text{by Associativity} \implies b = c\]
      Similar proof for left cancellation
      \newpage

    \textbf{Theorem 2.3 Uniqueness of inverses}: for each element $a \in G$, there is a unique element $b \in G$ such that $ab = ba = e$

      \textbf{Proof}: assume $b, c$ are both inverses of $a$. Then $ab = e$ and $ac = e$ so $ab = ac$

      Cancelling $a$ on both sides gives $b = c$

      \bigskip
      \textbf{Additional Notation}:
      \begin{itemize}
        \item $g^0 = e$
        \item typically do not allow noninteger exponents like $g^{1/2}$
        \item exponent addition and multiplication laws hold: $g^{m}g^{n} = g^{m + n}$ and $(g^m)^n = g^{mn}$
        \item exponent expansion of 2 elements typically does not hold: $(ab)^n \neq a^n b^n$
        \item because of uniqueness of inverse, for a valid group there is only 1 solution to $ax = b$, namely $a^{-1}$
      \end{itemize}
      \bigskip

    \textbf{Theorem 2.4 Socks-Shoes Property}: for elements $a,b$, $(ab)^{-1} = b^{-1}a^{-1}$

    \textbf{Proof}: $(ab)(b^{-1}a^{-1}) = a(bb^{-1}a^{-1})$ by Associativity $= aea^{-1} = aa^{-1} = e$.

    Thus $(ab)(ab)^{-1} = (ab)(b^{-1}a^{-1}) = e$ and $(ab)^{-1} = b^{-1}a^{-1}$
  \section{Finite Groups; Subgroups}
  \textbf{Order of a Group}: number of elements in a group, denoted $|G|$

  \textbf{Order of an Element}: smallest positive integer $n$ such that $g^n = e$, denoted $|g|$
  \begin{itemize}
    \item for additive notation, this would be $ng = 0$
    \item if no such integer exists, element $g$ has \textbf{infinite order}
  \end{itemize}

  \textbf{Examples}
  \begin{itemize}
    \item let $U(15) = \{1, 2, 4,7, 8, 11, 13, 14\}$ under multiplication mod $15$.
      \begin{itemize}
        \item the group has order $8$
        \item order of element $7$, $7^1 \equiv 7, 7^2 \equiv 4, 7^3 \equiv 13, 7^4 \equiv 1$ so $|7| = 4$
      \end{itemize}
    \item $Z$ under ordinary addition:
      \begin{itemize}
        \item every nonzero element has infinite order since the sequence $a, 2a, 3a, \ldots$ never includes $0$ when $a \neq 0$
      \end{itemize}
  \end{itemize}

  \textbf{Subgroup}: a subset $H$ of group $G$ that is a group under the operation of $G$, denoted $H \leq G$
  \begin{itemize}
    \item \textbf{Trivial Subgroup}: $\{e\}$ of $G$
    \item \textbf{Nontrivial Subgroup}: any subgroup that is not $\{e\}$
    \item Note: $Z_n$ under addition modulo $n$ is \textbf{not} a subgroup of $Z$ under addition since it isn't an operation under $Z$
  \end{itemize}
  \subsection{Subgroup Tests}
  \textbf{One-Step Subgroup Test}: let $G$ be a group and $H \subseteq G$ with $H \neq \emptyset$. If $(\forall a,b \in H)$$[ab^{-1} \in H]$, then $H \leq G$
  \begin{itemize}
    \item in additive notation, if $a-b \in H$ whenever $a,b \in H$ then $H \leq G$
  \end{itemize}
  \textbf{Proof}: 
  \begin{itemize}
    \item associativity: $H$ has the same operation as $G$
    \item identity: pick any $x \in H$ and let $a = b = x$, then $xx^{-1} = e \in H$
    \item inverse: pick any $x \in H$ and let $a = e, b = x$, then $ex^{-1} = x^{-1} \in H$
    \item closure: pick any $x, y \in H$ and let $a = x, b = y^{-1}$, then $xy = x(y^{-1})^{-1} \in H$
  \end{itemize}
  \bigskip

  \textbf{Steps to apply One-Step Subgroup Test}:
  \begin{enumerate}
    \item identify property $P$ that distinguishes elements of $H$ (defining condition)
    \item prove that the identity has property $P$ (verify $H$ is nonempty)
    \item assume elements $a, b$ have property $P$ and use assumption to show $ab^{-1}$ has property $P$
  \end{enumerate}

  \textbf{Example}: let $G$ be an Abelian group with identity $e$. Then $H = \{x \in G | x^2 = e\}$ is a subgroup of $G$
  \begin{itemize}
    \item defining property of $H$ is condititon $x^2 = e$
    \item $e^2 = e$ so $H$ is nonempty
    \item assuming $a, b \in H$, we have $a^2 = b^2 = e$
    \item since $G$ is Abelian, $(ab^{-1})^2 = ab^{-1}ab^{-1} = a^2(b^{-1})^2 = a^2(b^2)^{-1} = ee^{-1} = e$. Therefore $ab^{-1} \in H$
    \item so by One-Step Subgroup Test, $H \leq G$
  \end{itemize}

  \textbf{Example}: let $G$ be an Abelian group under multiplication with identity $e$, then $H = \{x^2 | x \in G\}$ is a subgroup of $G$
  \begin{itemize}
    \item since $e^2 = e$, identity has the correct from so $H$ is nonempty
    \item assuming $a^2, b^2 \in H$ and since $G$ is Abelian, we can write $a^2(b^2)^{-1}$ as $(ab^{-1})^2$ thus $H \leq G$
  \end{itemize}
  \bigskip

  \textbf{Two-Step Subgroup Test}: let $G$ be a group and $H \subseteq G$ with $H \neq \emptyset$. If $(\forall a, b \in H) [ab \in H \wedge a^{-1} \in H]$ then $H \leq G$

  \textbf{Proof}: given $a, b \in H$, since $b^{-1} \in H$, we have $ab^{-1} \in H$ so the One-Step Subgroup Test is satisfied

  \textbf{Example}: let $G$ be an Abelian group. Then $H = \{x \in G \, \mid \, |x| \text{ is finite }\}$ is a subggroup of $G$
  \begin{itemize}
    \item $e^1 = e$ so $H$ is non-empty
    \item assume $a, b \in H$ and let $|a| = m$ and $|b| = n$
    \item since $G$ is Abelian, we have $(ab)^{mn} = (a^m)^n(b^n)^m = e^ne^m = e$ so $ab$ has finite order
    \item $(a^{-1})^m = (a^m)^{-1} = e^{-1} = e$, so $a^{-1}$ has finite order
    \item by Two-Step Subgroup Test, $H \leq G$
  \end{itemize}
  \textbf{Example}: let $G$ be an Abelian group and $H, K$ be subgroups of $G$. Then $HK = \{hk | h \in H, k \in K\}$ is a subgroup of $G$
  \begin{itemize}
    \item $e = ee \in HK$
    \item suppose $a, b \in HK$. By definition of $H$ there are elements $h_1, h_2 \in H$ and $k_1, k_2 \in K$ such that $a = h_1k_1$ and $b = h_2k_2$
    \item to prove that $ab \in HK$, observe that since $G$ is Abelian and $H, K \leq G$, we have $ab = h_1k_1h_2k_2 = (h_1h_2)(k_1k_2) \in HK$
    \item likewise $a^{-1} = (h_1k_1)^{-1} = k_1^{-1}h_1^{-1} = h_1^{-1}k_1^{-1} \in HK$
    \item by Two-Step Subgroup Test, $HK \leq G$
  \end{itemize}
  To show a subset of a group is not a subgroup show that either
  \begin{itemize}
    \item identity is not in the set
    \item an element's inverse is not in the set
    \item 2 elements whose product is not in the set
  \end{itemize}
  \textbf{Example}: $G$ be a group of nonzero real numbers under multiplication. $H = \{x \in G | x = 1 \vee x \in I\}$ and $K = \{x \in G | x \geq q\}$.
  \begin{itemize}
    \item $H$ is not a subgroup of $G$ since $\sqrt{2} \cdot \sqrt{2} = 2 \notin H$
    \item $K$ is not a subgroup of $G$ since $2 \in K$ but $2^{-1} \notin K$
  \end{itemize}

  \textbf{Finite Subgroup Test}: let $G$ be a group and $H \subseteq G$ with $|H| < \infty$. If $(\forall a, b \in H)[ab \in H]$ then $H \leq G$

  \textbf{Proof}: need to show that $a^{-1} \in H$ for all $a \in H$ then apply Two-Step Subgroup Test
  \begin{itemize}
    \item given $a \in H$, if $a = e$ then $a^{-1} = e$
    \item if $a \neq e$, consider $S = \{a, a^1, \ldots\} \in H$ by closure. Since $H$ is finite 2 of elements, say $a^j = a^k$ for $1 \leq j < k$ must be identical. Simplifying we get $e = a^{k-j} = aa^{k-j-1}$ so $a^{k-j-1}$ is the inverse of $a$ and is in $H$
  \end{itemize}

  \subsection{Examples of Subgroups}
 \textbf{$\langle a \rangle$ is a Subgroup}: let $G$ be a group and let $a \in G$. Then $\langle a\rangle \leq G$

  \textbf{Proof}:
  \begin{itemize}
    \item since $a \in \langle a \rangle,$ the subset is not empty
    \item let $a^n, a^m \in \langle a \rangle$. Then $a^n(a^m)^{-1} = a^{n-m} \in \langle a \rangle$
    \item by One-Step Subgroup test, $\langle a \rangle \leq G$
  \end{itemize}

  \textbf{Note}: $\langle a \rangle$ is called the \textbf{cyclic subgroup} of $G$ generated by $a$
  \begin{itemize}
    \item $a^ia^j = a^{i+j} = a^{j+i} = a^ja^i$ so every cyclic group is Abelian
  \end{itemize}

  \bigskip

  \textbf{Center of a Group}: $Z(G)$ the center of group $G$ is the subset of elements in $G$ that commute with every element $\in G$
  \[(\forall x \in G) [Z(G) = \{a \in G | ax = xa\}]\]

  \textbf{Center is a Subgroup}: the center of $G$ is a subgroup of $G$

  \textbf{Proof}: assume $a, b \in Z(G)$ so for all $x \in G$ we have $ax = xa$ and $bx = xb$. Then use Two-Step Subgroup Test:
  \begin{itemize}
    \item since $xa = ax$ we have $a^{-1}xaa^{-1} = a^{-1}axa^{-1}$ so $a^{-1}x = xa^{-1}$ and $a^{-1} \in Z(G)$
    \item since $abx = axb = xab$, $ab \in Z(G)$
    \item by Two-Step Subgroup Test, $Z(G) \leq G$
  \end{itemize}

  \bigskip

  \textbf{Centralizer}: let $a$ be an element of $G$. The \textbf{centralizer} of $a \in G$, denoted $C(a)$, is the set of all elements in $G$ that commute with $a$
  \[C(a) = \{g \in G | ga = ag\}\]

  \textbf{C(a) is a Subgroup}: for each $a \in G$, the centralizer of $a$ is a subgroup of $G$

  \textbf{Proof}:
  \begin{itemize}
    \item $ae = a = ea$ so $e \in C(a)$ and is non-empty
    \item take any $x, y \in C(a)$, then $ax = xa$ and $ay = ya$. Then $(xy)a = x(ya) = x(ay) = (xa)y = (ax)y = a(xy)$ so $xy \in C(a)$
    \item take any $x \in C(a)$, then $ax = xa$. Then $x^{-1}a = x^{-1}ae = x^{-1}a(xx^{-1}) = x^{-1}(ax)x^{-1} = x^{-1}(xa)x^{-1} = eax^{-1} = ax^{-1}$ so $x^{-1} \in C(a)$
    \item By Two-Step Subgroup Test, $C(a) \leq G$
  \end{itemize}
  \section{Cyclic Groups}
  A group $G$ is \textbf{cyclic} if there is an element $a \in G$ such that $G = \{a^n \mid n \in Z\}$
  \begin{itemize}
    \item $a$ is called the \textbf{generator} of $G$
    \item cyclic group generated by $a$ is denoted $\langle a \rangle$
  \end{itemize}
  \textbf{Examples}:
  \begin{itemize}
    \item $U(10) = \{1, 3, 7, 9\} = \{3^0, 3^1, 3^3, 3^2\} = \langle 3 \rangle$. Similar for $\langle 7 \rangle$
    \item $U(8) = \{1, 3, 5, 7\}$ has no cyclic group
      \begin{itemize}
        \item $\langle 1 \rangle \rightarrow \{1\} \neq U(8)$
        \item $\langle 3 \rangle \rightarrow \{3, 1\} \neq U(8)$
        \item $\langle 5 \rangle \rightarrow \{5, 1\} \neq U(8)$
        \item $\langle 7 \rangle \rightarrow \{7, 1\} \neq U(8)$
      \end{itemize}
  \end{itemize}

  \textbf{Criterion for $\mathbf{a^i = a^j}$}: let $G$ be a group and $a \in G$
  \begin{itemize}
    \item if $a$ has infinite order, then $a^i = a^j$ if and only if $i = j$
    \item if $a$ has finite order $(n)$, then $\langle a \rangle = \{e, a, a^2, \ldots, a^{n-1}\}$ and $a^i = a^j$ if and only if $n$ divides $i -j$
  \end{itemize}

  \textbf{Proof}:
  \begin{itemize}
    \item if $a$ has infinite order, then there is no $n > 0$ such that $a^n = e$

      Since $a^i = a^j$, we have that $a^{i-j} = e$ and thus $i - j = 0$
    \item if $a$ has finite order $(n)$, let $a^k$ be an arbitrary member of $\langle a \rangle$

      By division algorithm, $k = qn + r$ for $q, r \in Z$ and $0 \leq r < n$
      So $a^k = a^{qn + r} = (a^n)^qa^r = a^r$

      Thus $a^k \in \{e, a, \ldots, a^{n-1}\}$ and $\langle a \rangle = \{e, a, \ldots, a^{n-1}\}$

      Next, assume $a^i = a^j$, which implies $a^{i-j} = e$

      By division algorithm, $i -j = qn + r$ for $q, r \in Z$ and $0 \leq r < n$

      Then $a^{i-j} = a^{qn + r}$ and $e = a^r$.

      Since $n$ is the least positive integer such that $a^n = e$, $r$ must be $0$

      Thus $n \mid i - j$

      Conversely, if $i - j = nq$, then $a^{i - j} = a^{nq} = e^q = e$ so $a^i = a^j$
  \end{itemize}

  \bigskip

  \textbf{Corollary 1}: for any $a \in G, |a| = |\langle a \rangle |$

  \textbf{Corollary 2}: let $a \in G$ with $|a| = n$. If $a^k = e$, then $n \mid k$

  \textbf{Proof}: since $a^k = e = a^0$, by the previous theorem/criterion, we know that $n \mid k - 0$
  
  \bigskip

  \textbf{Corollary 3}: if $a, b$ belong to a finite group and $ab = ba$, then $|ab|$ divides $|a||b|$

  \textbf{Proof}: let $|a| = m$ and $|b| = n$

  $(ab)^{mn} = (a^m)^n(b^n)^m = e$ so by the Corollary 2, $|ab|$ divides $mn$

    \bigskip

  \textbf{Theorem 4.2}: let $a \in G$ where $|a| = n$ and let $k$ be a positive integer then 
  \begin{itemize}
    \item $\langle a^k \rangle = \langle a^{\gcd(n, k)} \rangle$
    \item $|a^k| = n / \gcd(n, k) \rangle$
  \end{itemize}

  \textbf{Proof}: let $d = \gcd(n, k)$ and $k = dr$

  Since $a^k = (a^d)^r$, we have $\langle a^k \rangle \subseteq \langle a^d \rangle$ by closure

  Since gcd can be written as a linear combo, we have $d = ns + kt \implies a^d = a^{ns + kt} = (a^n)^s(a^k)^t = (a^k)^t \in \langle a^k \rangle$ so $\langle a^d \rangle \subseteq \langle a^k \rangle$

  Thus $\langle a^k \rangle = \langle a^d \rangle$

  \bigskip

  \textbf{Order of Elements in a finite Cyclic Group}: in a finite cyclic group, the order of elements divides the order of the group

  \textbf{Criterion for $\mathbf{\langle a^i \rangle = \langle a^j \rangle}$ and $\mathbf{|a^i| = |a^j|}$}: let $|a| = n$ then
  \begin{itemize}
    \item $\langle a^i \rangle = \langle a^j \rangle$ if and only if $\gcd(n, i) = \gcd(n, j)$
    \item $|a^i| = |a^j|$ if and only if $\gcd(n, i) = \gcd(n, j)$
  \end{itemize}

  \textbf{Generators of Finite Cyclic Groups}: let $|a| = n$ then
  \begin{itemize}
    \item $\langle a \rangle = \langle a^j \rangle$ if and only if $\gcd(n, j) = 1$
    \item $|a| = |\langle a^j \rangle |$ if an only if $\gcd(n, j) = 1$
  \end{itemize}

  \textbf{Generators of $Z_n$}: $k \in Z_n$ is a generator of $Z_n$ if and only if $\gcd(n, k) = 1$

  \bigskip

  \textbf{Fundamental Theorem of Cyclic Groups}: every subgroup of a cyclic group is cyclic. Moreover
  \begin{itemize}
    \item if $|\langle a \rangle | = n$, then the order of any subgroup of $\langle a \rangle$ is a divisor of $n$
    \item for each positive divisor $k$ of $n$, the group $\langle a \rangle$ has exactly 1 subgroup of order $k$, namely $\langle a^{n/k} \rangle$
  \end{itemize}

  \bigskip

  \textbf{Number of Elements of Each Order in a Cyclic Group}: if $d$ is a positive divisor of $n$, then the number of elements of order $d$ in a cyclic group of order $n$ is $\phi(d)$ (Euler phi function)

  \section{Permutation Group}
  \textbf{Permutation} of a set $A$ is a function from $A$ to $A$ that is both 1-1 and onto

  \textbf{Permutation Group} of set $A$ is a set of permutations of $A$ that form a group under function composition
\end{document}
